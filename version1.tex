% -- Encoding UTF-8 without BOM
% -- XeLaTeX => PDF (BIBER)

\documentclass[french]{cv-style}     % Add 'print' as an option into the square bracket to remove colours from this template for printing. 
                                      % Add 'espanol' as an option into the square bracket to change the date format of the Last Updated Text
%\setdefaultlanguage{french}
%\sethyphenation[variant=spanish]{}{}  % Add words between the {} to avoid them to be cut 

\usepackage{graphicx}

\begin{document}

\header{Maxence }{Grosjean}
\lastupdated

%----------------------------------------------------------------------------------------
% SIDEBAR SECTION  -- In the aside, each new line forces a line break
%----------------------------------------------------------------------------------------
\begin{aside}
\section{ab}
\includegraphics[width=4cm]{photo}
%
\section{Profil}
Né le 24/08/1995
12 Rue du 8 Mai 1945
69510 Soucieu-en-Jarrest
%
\section{Contact}
maxence.grosjean@cpe.fr
github.com/maxenceng
0618354310
%
\section{Langues}
\textbf{Anglais}  bilingue
\textbf{Japonais}  débutant
%
\section{langages de programmation}
JavaScript, PHP, Java
%
\section{Frameworks}
ExpressJS, ReactJS, Slim, AngularJS, JEE, Sass, KoaJS
%
\section{Autres outils}
Webpack, ESLint, SQL, Git, Vim
%
\end{aside}

\vspace{0.2cm}
\section{Expériences professionnelles}
\begin{entrylist}
%------------------------------------------------
\entry
  {\scalebox{.8}[1.0]{Février -- Août 2019}}
  {SOLUTEC}
  {69100 Villeurbanne}
  {\jobtitle{Stagiaire DevOps}\\
  Résumé travail SOLUTEC}
%------------------------------------------------
\entry
  {\scalebox{.8}[1.0]{Février -- Août 2018}}
  {SoLive}
  {75010 Paris}
  {\jobtitle{Développeur JavaScript}\\
  Résumé travail SoLive}
%------------------------------------------ ------
\entry
  {\scalebox{.8}[1.0]{Juillet -- Décembre 2017      }}
  {CommuterClub}
  {EC2A 3BA Londres, Angleterre}
  {\jobtitle{Développeur PHP}\\
  Résumé travail CC.\\
  \textbf{Proyectos desarrollados:}
  \begin{itemize}\small{
    \item Parseo de HTML, XML, CSV, JSON (por medio de la biblioteca Scrapy).
    \item Categorización del contenido a partir de sus descripciones\\
    (utilizando las bibliotecas NLTK).
    \item Paralelización de consultas a APIs, optimizando tiempos de respuesta\\
    (manejo de Threads).
    \item Desarrollo de APIs para back-end.}
  \end{itemize}}
%------------------------------------------------
\entry
  {\scalebox{.8}[1.0]{Juillet 2016}}
  {Ville de Lyon}
  {69008 Lyon}
  {\jobtitle{Création d'outils d'aide et simulations d'éclairement}\\
  Résumé travail VdL}
 
\end{entrylist}
%----------------------------------------------------------------------------------------
% EDUCATION SECTION
%----------------------------------------------------------------------------------------
\section{Éducation}
  \vspace{-0.2cm}
\begin{entrylist}
%------------------------------------------------
\entry
{\scalebox{.8}[1.0]{2015--2019}}
{CPE Lyon}
{Filière Sciences du numérique}
%------------------------------------------------
\entry
{\scalebox{.8}[1.0]{2013--2015}}
{Classes Préparatoires CPE Lyon}
%------------------------------------------------
% \entry
% {\scalebox{.8}[1.0]{2010--2013}}
% {Lycée Saint Thomas d'Aquin}

\end{entrylist}
%----------------------------------------------------------------------------------------
% AWARDS SECTION
%----------------------------------------------------------------------------------------
\section{Centres d'intérêts}
  \vspace{-0.2cm}
\begin{entrylist}
%------------------------------------------------
\entry
{Musculation}
%------------------------------------------------
\entry
{Cinéma}
%------------------------------------------------
\entry
{Jeux vidéo}
%------------------------------------------------
\entry
{Musique}
%------------------------------------------------
\end{entrylist}
  \vspace{-0.2cm}

\end{document}